\documentclass[]{article}
\usepackage{amsmath}
\usepackage{amsfonts}

%opening
\title{Notes on Jackson-like queue networks}
\author{Chase Dowling}

\begin{document}

\maketitle

\section{Notation}

Graph $G = (V,E)$, adjacency $A$, $n \times n$, in number of vertices. For vertices $i,j \in V$, we say $i$ is connected to $j$, $i\sim j$ if the ordered pair $(i,j) \in E$. Directionality is implied by the ordering.

\section{Jackson Networks}

A Jackson Network is a network of queues where exogenous jobs arrive according to a Poisson process (exponential interarrival) with rate $\lambda > 0$.  For a job completed at vertex $i$, jobs are routed to vertex $j$ with some probability. For vertices $j\sim i$, jobs move onto another vertex probability $\sum_{j} p_{ij}$ or leave the network with probability $1 - \sum_{j} p_{ij}$. 

The overall arrival rate at a vertex $i$, $\lambda_i$ becomes:

\[\lambda_i = \lambda p_{0i} + \sum_{j=1} \lambda_i p_{ji}\]

for $(j,i) \in E$, and $p_{0i}$ equal to the probability that an exogenous job arrives at $i$ first. Additionally, $p_{i0}$ is equal to the probability that a job item has left the network. The 0'th vertex acts as both source and sink--this is referred to as an \emph{open} Jackson network.

\section{Special Case}

We wish to examine a special case where if the queue at a vertex $i$ is full, a job is automatically rerouted to another queue. Once a job reaches the first available queue and the task is completed, the job leaves the network. Each queue in the network is an G/M/n/n queue. Note that although a vertex's exogenous arrivals may be Poissonian, it is not the case that the combination of inter-vertex arrivals plus exogenous arrivals is also Poissonian. 

Consider such a network of $k$ queues with uniform service rate $\mu$ and exogenous arrival rate $\lambda$. We can immediately observe that the expected number of tasks that have exited the system at time $t$ is bounded above by $(kn) \mu t$ (best possible service rate). The number of tasks within the system is at least $(\lambda - \mu)t$. How do I know all of this is true?

Def'n: Call a directed acyclic graph a chain if the first and last vertex has degree 1, all others have degree 2. The longest path in the chain is the graph itself.

Observe that a chain of queues with $p_01$ = 1, the steady state behavior of a chain of $k$ M/M/1 queues is equivalent to an M/M/k queue with capacity $k$. 

Claim: A tree of $k$ M/M/1 queues steady state service rate is bounded above by that of an M/M/n queue where $n$ is the length of the longest chain in the tree.

Proof: Moving down the tree of available queues cuts off branches of potentially available queues, congestion at root nodes causes the bounding behavior. How bad?

\section{Similarities to Jackson network}



\end{document}

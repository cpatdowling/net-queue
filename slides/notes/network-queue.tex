\documentclass[]{article}
\usepackage{amsmath}
\usepackage{amsfonts}

%opening
\title{Notes on Jackson-like queue networks}
\author{Chase Dowling}

\begin{document}

\maketitle

\section{Notation}

Graph $G = (V,E)$, adjacency $A$, $n \times n$, in number of vertices. For vertices $i,j \in V$, we say $i$ is connected to $j$, $i\sim j$ if the ordered pair $(i,j) \in E$. Directionality is implied by the ordering.

\section{Jackson Networks}

A Jackson Network is a network of queues where exogenous jobs arrive according to a Poisson process (exponential interarrival) with rate $\lambda > 0$.  For a job completed at vertex $i$, jobs are routed to vertex $j$ with some probability. For vertices $j\sim i$, jobs move onto another vertex probability $\sum_{j} p_{ij}$ or leave the network with probability $1 - \sum_{j} p_{ij}$. 

The overall arrival rate at a vertex $i$, $\lambda_i$ becomes:

\[\lambda_i = \lambda p_{0i} + \sum_{j=1} \lambda_i p_{ji}\]

for $(j,i) \in E$, and $p_{0i}$ equal to the probability that an exogenous job arrives at $i$ first. Additionally, $p_{i0}$ is equal to the probability that a job item has left the network. The 0'th vertex acts as both source and sink--this is referred to as an \emph{open} Jackson network.

\section{Parking Garage Problem}

Consider an $M/M/1/1$ queue with Poisson arrival $\lambda$ and service rate $\mu$ (representative of a blockface with 1 parking space). If the single server is busy, any new arrival is automatically rejected (and elect to park in a parking garage). We can observe this is a birth-death process with two states for the queue: busy or idle. Denote these states as either $X_0$ or $X_1$.

We wish to obtain the transition probabilities for this continuous time Markov Chain for which all states communicate. Using the Kolmogorov Backwards Equations, we obtain the following system of differential equations:

\begin{equation}
p'_{00} = \lambda p_{10}(t) - \lambda p_{00}(t) ]
\end{equation}
\begin{equation}
p'_{10} = \mu p_{00}(t) - \mu p_{10}(t)
\end{equation}

Multiplying (1) by $\mu$, (2) by $\lambda$, adding, and integrating the resulting equation we obtain:

\begin{equation}
\mu p_{00}(t) + \lambda p_{10}(t) = c
\end{equation}

To solve for the constant of integration, we assume the initial condition that $p_{00}(0) = 1$, thus $p_{10}(0) = 0$, and $c = \mu$. Substituting (3) into (1):

\begin{equation}
p'_{00}(t) = \mu - (\mu + \lambda) p_00(t)
\end{equation}

Let $h(t) = p_00(t) - \frac{\mu}{\mu + \lambda}$. Then $h'(t) = p'_{00}(t) = -(\lambda + \mu)h(t)$

\begin{align}
\frac{h'(t)}{h(t)} &=\,\, -(\mu + \lambda) \\
\int \frac{h'(t)}{h(t)} &=\,\, \int -(\mu + \lambda) dt \\
log \,\, h(t) &=\,\, -(\mu + \lambda)t + c \\
h(t) &=\,\, k e^{-(\mu + \lambda)t} \\
p_{00}(t) &=\,\, k e^{-(\mu + \lambda)t} + \frac{\mu}{\lambda + \mu} \\
\end{align}

Using $p_{00}(0) = 1$, $k = \frac{\lambda}{\mu + \lambda}$, therefore

\begin{equation}
p_{00}(t)  = \frac{\lambda}{\mu+\lambda} e^{-(\mu + \lambda)t} + \frac{\mu}{\mu + \lambda}
\end{equation}
\begin{equation}
p_{01}(t) = (1 - p_{00}(t))
\end{equation}

What is the distribution of $P_{01}$ and $P_{10}$? Is the expectation of the convolution to the expected transit time for vehicle?

\section{Recovering Steady State}

Burke's theorem states that for such an M/M/1 queue in the steady state with arrivals a Poisson process with rate $\lambda$, the departure process is also a Poisson process with rate $\lambda$. To evaluate the validity of this claim for a finite state birth-death process, we must determine the limiting probabilities from the above section. The existence of the limiting probabilities requires the assumptions

\begin{itemize}
\item[a]all states of the Markov chain communicate
\item[b]the expected time to return to any state is finite
\end{itemize}

(a) is already true by assumption, and (b) is true for finite service rates and all reasonable distributions of $P_{01}$ and $P_{10}$.

\begin{equation}
P_j := \lim_{t \rightarrow \infty} p_{ij}(t)
\end{equation}

Consider the Kolmogorov Forward Equations in the limit:

\[p'_{ij}(t) = \sum_{k \neq j} q_{kj} p_{ik}(t) - v_{j}p_{ij}(t) \]

\[\lim_{t \rightarrow \infty} p'_{ij}(t) = \lim_{t \rightarrow \infty} \sum_{k \neq j} q_{kj} p_{ik}(t) - v_{j}p_{ij}(t) \]

If (7) converges, then $\lim_{t \rightarrow \infty} p'_{ij}(t)$ must converge to 0, so:

\[0 = \sum_{k \neq j} q_{kj} P_k - v_{j}P_j \]

Up to this point no assumptions are made about the nature of the distributions of the $p_{ij}$'s

For any birth-death process with exit rates $\lambda_i$, enter rates $\mu_i$, assuming the limiting probabilities exist, solving for the sum of the balance equations in terms of $P_{0}$

\begin{equation}
1 = P_0 + P_0 \sum_{i=0}^{n} \frac{\lambda_{i-1} \ldots \lambda_{0}}{\mu_i \ldots \mu_0}
\end{equation}

The necessary and sufficient condition then for the existence of the limiting probabilities is that the sum in (8) is bounded. For finite states $n$ this is true for reasonable values of $\lambda < \infty$ and $\mu > 0$. For infinite states, $\lambda < \mu$ is also required. In our simple, 2-state case, we have that:

\begin{equation}
P_0 = \frac{1}{1 + \lambda/\mu} 
\end{equation}
\begin{equation}
P_1 = \frac{\lambda/\mu}{1 + \lambda/\mu}
\end{equation}

In the limiting case, we can recover the number of vehicles then served, and therefore, the number of vehicles rejected and sent to a parking garage.

\section{Rejection}

We're ultimately concerned with the expected time between cars being sent to a parking garage. As $\lambda >> \mu$, $P_1 \rightarrow 1$, w.r.t. the aim of determining traffic flow between queues based on rejection.

Let the inter arrival times $_i$ be exponentially distributed with parameter $\lambda$, $S(t)$ be the corresponding Poisson arrival process, and let $E[R] :=$ \emph{expected time between rejections}. Because arrivals are independent of the state of queue, we can observe that $E[R|queue full] = 1/\lambda$.

\begin{align}
E[R] &= \int_0^{\infty} E[R | S(t_0) = 1] S(t) dt \\
\, &= \int_0^{\infty}E[R | S(t_0) = 1, X(t_0) = X_0] P_0(t) S(t) dt + \int_0^{\infty}E[R | S(t_0) = 1, X(t_0) = X_1] P_1(t) S(t) dt \\
\, &= (\frac{1}{\lambda} + E[R])P_0 + \frac{1}{\lambda} P_1 \\
\, &= \frac{1}{\lambda} P_0 + \frac{1}{\lambda} P_1 + P_0 E[R] \\
E[R](1 - P_0) &=  \frac{1}{\lambda} P_0 + \frac{1}{\lambda} (1 - P_0) \\
E[R] &= \frac{1}{\lambda}\frac{1}{1- P_0} \\
\, &= \frac{\mu + \lambda}{\lambda^2}
\end{align} 


%\section{Special Case}

%We wish to examine a special case where if the queue at a vertex $i$ is full, a job is automatically rerouted to another queue. Once a job reaches the first available queue and the task is completed, the job leaves the network. Each queue in the network is an G/M/n/n queue. Note that although a vertex's exogenous arrivals may be Poissonian, it is not the case that the combination of inter-vertex arrivals plus exogenous arrivals is also Poissonian. 

%Consider such a network of $k$ queues with uniform service rate $\mu$ and exogenous arrival rate $\lambda$. We can immediately observe that the expected number of tasks that have exited the system at time $t$ is bounded above by $(kn) \mu t$ (best possible service rate). The number of tasks within the system is at least $(\lambda - \mu)t$. How do I know all of this is true?

%Def'n: Call a directed acyclic graph a chain if the first and last vertex has degree 1, all others have degree 2. The longest path in the chain is the graph itself.

%Observe that a chain of queues with $p_01$ = 1, the steady state behavior of a chain of $k$ M/M/1 queues is equivalent to an M/M/k queue with capacity $k$. 

%Claim: A tree of $k$ M/M/1 queues steady state service rate is bounded above by that of an M/M/n queue where $n$ is the length of the longest chain in the tree.

%Proof: Moving down the tree of available queues cuts off branches of potentially available queues, congestion at root nodes causes the bounding behavior. How bad?

%\section{Similarities to Jackson network}



\end{document}
